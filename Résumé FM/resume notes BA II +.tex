\documentclass[11pt,french]{report}
  \usepackage{babel}
  \usepackage[utf8]{inputenc}   % LaTeX
  \usepackage[T1]{fontenc}      % LaTeX
	\usepackage{icomma}
	\usepackage{fontenc}
  \usepackage[colorlinks]{hyperref}

  \frenchbsetup{ItemLabeli==$>$}

% déclaration de formule pour annuité
\DeclareRobustCommand{\annuity}[1]{%
\def\arraystretch{0}%
\setlength\arraycolsep{.7pt}%
\setlength\arrayrulewidth{.3pt}% 
\begin{array}[b]{@{}c|}\hline
\\[\arraycolsep]%
\scriptstyle #1%
\end{array}%
}

% exemple d'utilisation de la fonction annuity
% \ddot{a}_{\annuity{n}i} 

\title{Notes calculatrice BA II +}
\author{David Beauchemin}

\begin{document}

\maketitle

\tableofcontents

\chapter{Notes supplémentaires}

\begin{enumerate}
\item Boverman offre aussi un \emph{PDF} d'explication, le document est disponible sur le Google Drive du groupe d'échange de document dans la section du cours de \href{https://drive.google.com/open?id=0B6kXivc6X9LITmdBVFVWSDAxeE0}{mathématique financière}.
\item J'ai mis aussi des résumés de formule du cours précédent et de la \emph{SOA} qui sont disponible dans le fichier \emph{ZIP}  \href{https://drive.google.com/open?id=0B6kXivc6X9LIYUFKaHcxNmtiUUk}{\emph{FM}}.
\item \textbf{Notes sur la légende d'écriture} : 
Un symbole + signifie prochaine touche à \textit{cliquer} est la suivante.

\end{enumerate}

\chapter{Format d'affichage, valeur future, valeur actualisée et taux nominaux}

\textit{*Chapitre 1 dans le livre}

\section{Format d'affichage}
\fbox{2ND} + \fbox{format} + nombre de décimale + \fbox{enter}

\section{Valeur future}
\begin{enumerate}

\item Accumulation simple
\\ (1 + taux d'intérêt) + \fbox{$ y^{x} $} + valeur de l'exposant (x) + \fbox{X} + montant à accumulé + \fbox{=}
\item Fonction TVM
\\ Légende :

\begin{enumerate}
\item \fbox{N} période ;
\item \fbox{I$ / $Y} taux d'intérêt par période ;
\item \fbox{PV} Valeur présente ;
\item \fbox{PMT} Paiement (annuité) ;
\item \fbox{FV} Valeur accumulée ;
\item Astuce : La fréquence du taux d'intérêt peut-être modifié. On pourrait mettre le taux annuel effectif et jouer avec les paramètres de la calculatrice pour avoir un taux d'intérêt mensuel.
\\ Voici comment, \fbox{I$ / $Y} et régler à 12 pour avoir un mensuel. De base, pour ne pas faire d'erreur laisser à 1. Mais toujours utile de savoir cette fonction.
\end{enumerate}

\item \textbf{Comment utilisé TVM :}
\\ Nombre prériode + \fbox{N} + taux d'intérêt + \fbox{I$ / $Y} + valeur à accumulé + \fbox{+$ / $-} + \fbox{PV} + \fbox{CPT} + \fbox{FV}
\\ 
\item \textbf{Astuce :} Pour afficher la valeur d'un des paramètres utilisé dans TVM, \fbox{RCL} + \fbox{N} ou \fbox{PV}... 
\\
\item \textbf{Astuce :}
\begin{LARGE}
Ne pas oublier de \textit{clear} les valeurs!!
\end{LARGE}
\\ \fbox{2ND} + \fbox{CLR TVM}

\end{enumerate}

\section{Trouver le taux d'intérêt}
Nombre de période + \fbox{N} + montant à accumuler + \fbox{PV} + montant future + \fbox{FV} + \fbox{CPT} + \fbox{I$ / $Y }

\section{Trouver le nombre de période}
Taux d'intérêt + \fbox{I$ / $Y } + valeur présente + \fbox{+$ / $- } + \fbox{PV} + montant future + \fbox{FV} + \fbox{CPT} + \fbox{N}

\section{Taux nominal et TVM}
Comme les taux nominaux sont divise par le nombre de période, on peut simplement faire : \\
Nombre prériode + \fbox{N} + ($i^{(m)} \div m$) + \fbox{=} + \fbox{I$ / $Y} + valeur à accumulé + \fbox{+$ / $-} + \fbox{PV} + \fbox{CPT} + \fbox{FV}
\\

\section{Taux équivalent}

Convertir un taux nominal en effectif : 

\fbox{2ND} + \fbox{ICONV} + \emph{NOM} (taux nominal) + \fbox{ENTER} + \fbox{$\Downarrow$} jusqu'à \emph{$C / Y$} (nombre de période) + \fbox{ENTER} + \fbox{$\Uparrow$} jusqu'à \emph{EFF} + \fbox{CPT}
\\
\\ Pour trouver un taux nominal on \fbox{CPT} \emph{NOM} et on fixe le taux effectif dans \fbox{EFF}.
\\
\\ Pour trouver un taux d'escompte, convertir \emph{d} en \emph{i}.

\end{document}
